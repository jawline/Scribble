\documentclass[]{final_report}
\usepackage{graphicx}
\usepackage{hyperref}


%%%%%%%%%%%%%%%%%%%%%%
%%% Input project details
\def\studentname{Blake Loring}
\def\reportyear{2013}
\def\projecttitle{Scribble Programming Language}
\def\supervisorname{Adrian Johnstone}
\def\degree{BSc (Hons) in Computer Science}
\def\fullOrHalfUnit{Full Unit} % indicate if you are doing the project as a Full Unit or Half Unit
\def\finalOrInterim{Interim Report} % indicate if this document is your Final Report or Interim Report

\begin{document}

\chapter{Overview}

\chapter{Language Examples}

\section{Hello World}

\begin{verbatim}

//Import the console package allowing us to write to stdout
package console := import("console");

func main() {
 console.Log("Hello World\n");
}
\end{verbatim}

\section{Variables}
\section{Loops}
\section{Functions}
\section{Arrays}
\section{Structures}

\chapter{Lexical}

\section{Identifiers}

In Scribble the lexical analyser defined an identifier (or ID) as one or more characters, underscores and digits starting with a character that is not a digit. It is defined by the regular expression
\begin{verbatim}
id [_|a-z|A-Z][a-z|A-Z|0-9|_]*
\end{verbatim}

\section{Value Constants}

There are several constant values which are identified by the lexical analyser.

\subsection{Integer}

In Scribble a integer constant is a string of digits with no decimal place, prefix or suffix. it is defined in Scribble by the regular expression 

\begin{verbatim}
digit [0-9]
integer {digit}+
\end{verbatim}

which will accept strings of the digits 0-9.

For example, 5 would be identified an integer but 5f, 5.0 or i5 would not be.

\subsection{Floats}

A float constant is a string of digits, optionally followed up with a second string of digits with the character 'f' as a suffix to identify it from other types which have decimal places. It is defined by the regular expression
\begin{verbatim}
real {integer}("."{integer})*
float {real}f
\end{verbatim}

For example 5f and 5.43f would be examples of floating point constants however 5.0, 5 or f5 would not be.

\subsection{Boolean}

A boolean is defined in Scribble by the two keywords true or false.

\subsection{String}

A string is a string of characters within two sets of '"'. It is defined by the regular expression \begin{verbatim}
string \"[^\"\n]*\"
\end{verbatim}.

\section{Operators List}

\begin{verbatim}
"+" - PLUS;
"-" - MINUS;
"*" - TIMES;
"/" - DIVIDE;

":=" - ASSIGN;
"=" - EQUALS;

">" - GREATER;
"<" - LESSER;

"." - LINK;
"->" - POINT;

"++" - INCREMENT;
"--" - DECREMENT;
\end{verbatim}

\section{Comments}

Scribble uses the //Comment Line and /* Block of text */ notation for comments. 

This means that when you write // the rest of the current line will be seen as a comment and ignored. This is how single line comments are achieved.

It also means that anything in between /* and */ will be ignored, allowing multi line comments.

\section{Keywords List}

\begin{verbatim}

"and" - AND;
"or" - OR;
"package" - PACKAGE;
"then" - THEN;
"if" - IF;
"else" - ELSE;
"struct" - STRUCT;
"func" - FUNCTION;
"for" - FOR;
"var" - VARIABLE;
"int" - TYPE_INT;
"bool" - TYPE_BOOL;
"float32" - TYPE_FLOAT32;
"nil" - NIL;
"string" - TYPE_STRING;
"void" - TYPE_VOID;
"return" - RETURN;
"while" - WHILE;
"import" - IMPORT;
"true" - TRUE;
"false" - FALSE;
"type" - TYPE;
"do" - DO;
"array" - TYPE_ARRAY;
"len" - LENGTH;

\end{verbatim}

\chapter{Identifiers}

An identifier in Scribble is either the local package name, the name of a function, a type or a variable.

No two identifiers should have the same name ( Regardless of what they are identifying or their scope ) and this will raise a parsing error.

Identifiers are always local to the package and are not accessible from other packages except for functions and types using an explicit syntax.

\subsection{Scope}

Scribble currently has a simple very scoping system.

Package, function and type identifiers are all global and their names are shared across all functions within a package.

Variable names are always local to the function that they are defined in from that line onwards. Variable redefinition is not allowed and statement blocks ( Such as declaring it inside a flow control statement ) will not effect this.

For example:
\begin{verbatim}
func main() {
 var j := 15;
 
 if true then {
 	j := 30;
 }

}
\end{verbatim}
and
\begin{verbatim}
func main() {
 
 if true then {
 	var j := 30;
 }
 
 j := 10;
}
\end{verbatim}
are valid, however
\begin{verbatim}
func main() {
 var j := 15;
 
 if true then { 
  var j := false; 
 }

}
\end{verbatim}
would cause a parsing error.

\chapter{Types}

Scribble is a strictly typed language. This means that every variable, function and value has a specific type. Unlike other language Scribble will never implicitly cast another object. A programmer will have to use one of the casting functions provided or write their own to cast between types, even the basic primitive types such as integers and floats.

For example
\begin{verbatim}
var j : int := 5;
var k : float32 := j;
\end{verbatim}
would cause a parsing error but
\begin{verbatim}
var j : int := 5;
var k : float32 := sys.Float32(j);
\end{verbatim}
would be correct.


Scribbles type system is right handed. This means that you declare the type of a variable, argument or function on the right hand side of its deceleration instead of the left.

If a function is not given a type then it is assumed to be of type void. You cannot explicitly create a function or variable of void type.

Examples:
\begin{verbatim}

//Void function which takes an argument of type int
func vFunction(j : int) {
 //Do something with j
}

//Function that returns an int
func iFunction() : int {
 var j : int := 15;
 return j;
}

//Main function which creates some variables
func main() {
 var j := iFunction();
 vFunction(j);
}
\end{verbatim}

\section{Primitive Types}

There are several primitive types in Scribble. Primitive types are the building blocks of a program, representing fixed size pieces of data within the language giving them a fixed range.

Primitive types are passed around as values and not references, this means that if you create a primitive variable like var j : int and pass it to a function which modifies the copy of j in the calling function will not be affected.

For example \begin{verbatim}
func Hello() {
 var j := 10;
 World(j);
 //Here j = 10 would still be true.
}

func World(x : int) {
 x := 12;
}
\end{verbatim}

This difference is the primary difference between referenced values, where a change will affect the value anywhere where the reference exists.

\subsection{bool}

A boolean value is capable of representing one of two values, true or false. They are identified with the bool keyword.

Examples:
\begin{verbatim}
var j : bool := false;
var r : bool := true;
\end{verbatim}

\subsection{int}

The int type represents a 32 bit signed integer value which is capable of representing integers between -2147483648 and 2147483647.

They are identified by the keyword int. 

\subsection{float32}

The float32 type represents 32 bit floating point numbers allowing approximations of numbers with decimal points to be used in a program. They are identified by the float32 keyword.

\section{Reference Types}

A reference type are types of a variable size which are stored on the heap, an area in memory which allows for dynamic memory allocation.

Reference types are garbage collected, this means that like Java the data is automatically destructed when a reference to it no longer exists.

The values (arrays, structures or strings ) which are referenced only exist once in memory and any change to these values will be reflected across any functions which have a reference to it. The references themselves are like C pointers and are local to the function.

For example
\begin{verbatim}
func ModifyArray(arr : array(int)) {

 if len(arr) > 0 then {
  arr[0] := 5;
 }

}

func DoStuff() {
 var j := [10]int;
 ModifyArray(j);
}
\end{verbatim}
In this case the value of j[0] after ModifyArray has been called would be 5 and not the initial value 0.

However in the case
\begin{verbatim}
func ModifyArray(arr : array(int)) {

 //Create a new array and assign it to arr
 arr := [10]int;

 if len(arr) > 0 then {
  arr[0] := 5;
 }

}

func DoStuff() {
 var j := [10]int;
 ModifyArray(j);
}
\end{verbatim}
the value of j[0] in DoStuff would remain unchanged, as the reference arr in the ModifyArray function was modified to point to a new array rather than the one referenced by j in DoStuff.

\subsection{string}

The string type represents a string of characters. It is really syntax sugar for an array of characters. When a string constant is written in Scribble the expression generated will be to create a new item on he heap and then load that data into it, so "Hello World" would make a new array of 12 characters ( Including the null terminator ) and then load the string into it as initial values.

Strings are immutable. Any operation on them using the functions provided will not change the memory of the string and will instead create a new string with which to place the result of the operations, this can have some performance implications as repeated string manipulations cause the heap and garbage collector to perform a lot of expensive operations.

The string keyword identifies this type.

\subsection{Arrays}

Arrays are capable of contain a set number of values of the same type, including reference types. Two arrays which represent different types are not equivalent ( So you couldn't assign an array of integers to a variable which is mean to store an array of booleans ).

Arrays are written in Scribble as array(subType), this differs from the subtype[] syntax often used in other languages. The changes where made in order to make type information clearer especially when dealing with a right handed type system.

The length of an array can also be obtained after it is declared by writing len(arrayValue). This will return the number of elements in the array, not the size in bytes.

\subsection{Structures}

Structures are a predefined collection of fields with specific types.

\section{Type Inference}

Scribble allows for a variables type to be inferred when it is declared instead of explicitly typed. This reduces some of the typed overhead caused by strict languages without allowing the issues which lazy typing systems can cause ( Such as being unable to easily identify the type of a value as well as errors caused by casting ).

The type is inferred from the type of the expression that it is being assigned to. The type of the expression is identified by the recursive inspection of the generated statement tree. Each element on the trees type is inferred from the type of its children down to the basic constant values.

For example the expression 5 + 5 will be inferred to be of type integer as the two constant values are of type int and the addition expression will produce a value of the same type as the values it is adding, additionally 5f + 5f would be inferred to be float32.

If the type of an expression is inferred to be 'Void' ( Scribbles keyword for expressions or functions which have no type ) then it cannot be used in variable type inference and a parser error will be raised.

\section{Types in the grammar}

The types defined above are expressed in the grammar as
\begin{verbatim}
Type: TYPE_INT {
	} | TYPE_STRING {
	} | TYPE_FLOAT32 {
	} | TYPE_BOOL {
	} | TYPE_ARRAY LPAREN Type RPAREN {
	/* Structure types name from this and other packages */	
	} | WORD {
	} | WORD LINK WORD {
	}
;

/**
 * BaseStructureInfo is the definition of each field within a structure in the form Name : Type.
 * Accepts 1 or more definition.
 */

BaseStructureInfo: WORD COLON Type {		
	} | BaseStructureInfo COMMA WORD COLON Type {
	}
;

/**
 * Defines a structure within a package
 */
 
Program TYPE WORD ASSIGN STRUCT LBRACKET BaseStructureInfo RBRACKET {
}

\end{verbatim}

\chapter{Operators \& Expressions}

\section{Constant Expressions}

Constant expressions in Scribble are the basic building blocks of the language. Allowing you to enter constant values into the language so that they can be used in calculations or assigned to variables.

\subsection{Booleans}

A boolean constant is expressed using the keywords 'true' or 'false'.

It is defined in the grammar as
\begin{verbatim}
TRUE {
} | FALSE {
}\end{verbatim}

\subsection{Integers}

A integer constant is expressed whenever the lexical analyser recognizes an integer using the regular expression defined in the lexical analysis chapter.

It is defined in the grammar as
\begin{verbatim}
INT {
}
\end{verbatim}

\subsection{Floats}

A floating point constant is expressed whenever the lexical analyser recognizes an float using the regular expression defined in the lexical analysis chapter.

It is defined in the grammar as
\begin{verbatim}
FLOAT32 {
}
\end{verbatim}

\subsection{Strings}

A string constant is expressed whenever the lexical analyser recognizes a string using the regular expression defined in the lexical analysis chapter.

As string a not a primitive type when a string constant expression is executed it will create a new string on the heap and return a reference to it, this differs from all of the other constant types which are all primitives and have no effect on the heap.

\begin{verbatim}
STRING {
}
\end{verbatim}

\section{Variable Definition}

A variable can be defined in Scribble either with an explicit type or with its type inferred from its initial assignment.

variable definition is defined in the grammar as
\begin{verbatim}
Variable {
} | Variable ASSIGN Expression {
} | VARIABLE WORD ASSIGN Expression {
}
\end{verbatim}

\section{Variable Expressions}

A variable expression will evaluate to the value of the specified variable.

This is defined in the grammar as \begin{verbatim}
ID {
}
\end{verbatim}

\section{Array Construction}

Arrays in Scribble are constructed using the [ArraySize]type notation. The initial value will be set to zero for numeric types, false for booleans and null for reference types.

Examples:
[100]int will construct an array of 100 integers.
[100]array(int) will construct an array of 100 integer arrays.

It is defined in the grammar as
\begin{verbatim}
LSQBRACKET Expression RSQBRACKET Type {
}
\end{verbatim}

\section{Structure Construction}

Structures are constructed using the StructureTypeName { Initial Values seperated by a comma } notation. Currently every field in a structure has to be given an initial value and they are assigned in the order that they are defined in the structure definition.

Examples:

type Hello { 
	name : string,
	age : int
};

var J := Hello{ "Bobby", 18 };
Will create a variable J which is an instance of the structure Hello with the name "Bobby" and the age 18.

It is defined in the grammar as
\begin{verbatim}
Arguments: {
} | Arguments_2 {
};

Arguments_2: Expression {
} | Arguments COMMA Expression {
};

Type LBRACKET Arguments RBRACKET {
}
\end{verbatim}

\section{Array Index Expression}

An array index expression evaluates to the value of an array at a given index written in the notation ArrayExpression[IndexExpression].

This is defined in the grammar as \begin{verbatim}
Expression LSQBRACKET Expression RSQBRACKET {
}
\end{verbatim}

\section{Structure Field Expression}

A structure field expression evaluates to a specified field within a given Structure expression ( An expression which evaluates to the type struct(Something))

It is defined in the grammar as
\begin{verbatim}
Expression '->' ID {
}
\end{verbatim}

\section{Operators}

\subsection{Arithmetic}

There are four basic arithmetic operators in Scribble. These are addition '+', subtraction '-', multiplication '*' and division '/'. These operators take a left and right numeric expression of the same type and perform the appropriate operation on it.

Examples: 
5 + 7 * 2
4 / 2 * 8

If the expressions are not numeric or not of the same type then a parsing exception will occur. For example 5 * 5 would be valid but "Hello" + "World" or 5f + 5 would not be.

It is defined in the syntax as
\begin{verbatim}
Expression PLUS Expression {
} | Expression MINUS Expression {
} | Expression TIMES Expression {
} | Expression DIVIDE Expression {
}
\end{verbatim}

\subsubsection{Minus Expression}

In addition to these arithmetic operators -Expression is also allowed to allow negative expressions like -5 or -x to be constructed. It is the same as writing 0 - 5 or 0 - x and is valid for any numeric type. It is defined in the grammar as 

\begin{verbatim}
MINUS Expression {
}
\end{verbatim}

\subsection{Increment \& Decrement}

In addition to the basic arithmetic operators you can also use the '++' and '--' operators to increment or decrement a integer variable by 1. An increment or decrement expression also returns the value of the variable. If the operator is placed before the variable name then it will return the value of the variable after it is modified and if it is placed after the id of the variable then it will return the value of the variable before it is modified.

Examples:
var i := 0;
i++ = 0 is true

var j := 0;
++i = 1 is true

These operators are defined in the grammar as
\begin{verbatim}
ID INCREMENT {
} | INCREMENT ID {
} | ID DECREMENT {
} | DECREMENT ID {
}
\end{verbatim}

\subsection{Comparison}

There are six basic comparison operators in Scribble. Equal '=', Not equal '!=', greater than '>', less than '<', less than or equal to '<=' and greater than or equal to '>='. These operators allow a comparison between two expressions. All primitive types can be compared using the = and != operators however only numerical types can be compared using the <, >, <=, >= operators.

One difference from C and many of the languages based of it is that assignment is done using the ':=' operator and a single '=' is now used for comparison instead of '=='.

These operators are only capable of comparing expressions of the same type. Mixing two different types, even numerical types, will cause a parsing error.

They are defined in the grammar as \begin{verbatim}
Expression EQUALS Expression {
} | Expression NOT EQUALS Expression {
} | Expression GREATER Expression {
} | Expression LESSER Expression {
} | Expression LESSER EQUALS Expression {
} | Expression GREATER EQUALS Expression {
}
\end{verbatim}

\subsection{Logical Operators}

And and Or are the two currently supported logical operators in Scribble. These operators take two boolean expressions and return a boolean value of the logical and or or. They are written using the 'and' and 'or' keywords instead of the traditional '\&' and '|'. This was done in an effort to make blocks of code more readable, especially since Scribble does not require parenthesis around flow control statements.

The and keyword will only execute the expression on its right hand side if the value on the left hand side is true.

Examples:
true or false, true or true, false or true will all return true.
false or false will return false.
true and true will return true.
true and false, false and false, false and true will all return false.
5 = 1 and 6 = 2 - The right hand side expression would never be checked because 5 does not equal 1.

It is defined in the grammar as
\begin{verbatim}
Expression AND Expression {
} | Expression OR Expression {
}
\end{verbatim}

\subsection{Assignment}

Assignment in Scribble is done using the ':=' operator. It takes a variable ID and an expression and sets the value of that variable to the value given when the expression is evaluated.

Assignment will only take place of the expression to be assigned is exactly the same type as the variable being assigned to. Any other values will have to be cast using a custom or built in function.

Assignments can be done on the index's of an array by using the Expression[IndexExpression] := Expression notation.

Assignment can also be done on the fields of a structure using the Structure Expression -> FieldId := Expression notation. 

Examples: \begin{verbatim}
var j := 65;
j := 192;
\end{verbatim}

Would create a variable j, set j's value to 65 and then set j's value to 192.

\begin{verbatim}
var arr := [10]int;
arr[0] := 5;
\end{verbatim}

Would create the array arr with 10 integers and then assign arr's 0th index to the value 5.

\begin{verbatim}
type J := struct {
	name : string
}

var j := J{"Hello"};
j->name := "Hello World";
\end{verbatim}
Will result in a structure of type J being created with the name field set to "Hello" and then the name field of j being set to "Hello World"

If the variable and expression types differ then a parsing error will occur.

Assignment is defined in the grammar as
\begin{verbatim}
LPAREN Expression RPAREN {
} | ID  Expression {
} | Expression '->' ID ':=' Expression {
} | Expression LSQBRACKET Expression RSQBRACKET ASSIGN Expression {
} 
\end{verbatim}

\section{Array Length Expression}

The length of an array can be obtained by using the len(ArrayExpression) notation.

examples:
len([100]int) = 100 and len([50]int) = 50

\begin{verbatim}LENGTH LPAREN Expression RPAREN {
}
\end{verbatim}

\section{Parenthesis Expressions}

Parenthesis to control the order in which expressions are evaluated. This is helpful in large or complex expressions.

Examples: (5 + 5) * (5 + 5) would evaluate to 10 * 10 which would be 100. 5 + 5 * 5 + 5 would evaluate to 10 * 5 + 5 which would be 55.

defined in the grammar as
\begin{verbatim}
'(' Expression ')' {
}
\end{verbatim}

\section{Function Call Expressions}

A function call is written as FunctionName(Arguments) or PackageName.FunctionName(Arguments). The type of a function call is the type of the function that is being called, for example sys.String(15) is of type string and sys.Log("Hello World"); is of type void.

Examples:

sys.Log(sys.String(15)); will print the number 15 to the screen.

sys.Pow(2, 4); will return 16.

defined in the grammar as
\begin{verbatim}
Arguments: {
} | Arguments_2 {
};

Arguments_2: Expression {
} | Arguments COMMA Expression {
};

FunctionCall: WORD LPAREN Arguments RPAREN {
} | WORD LINK WORD LPAREN Arguments RPAREN {
};
\end{verbatim}

\chapter{Statements}

\section{Flow Control}

There are three flow control structures in Scribble.

\subsection{If}

The If statement will execute the code in its body if the given boolean expression evaluates to true. You can also supply an optional set of code to be execute if the statement evaluates to false.

Syntax: 
\begin{verbatim}
if Expression then {
	Code
}

or

if Expression then {
	Code
} else {
	Code
}
\end{verbatim}

Is is defined in the grammar by
\begin{verbatim}
IfStatements: 
Statement {
} | LBRACKET Statements RBRACKET {
}

IF Expression THEN IfStatements  {
} | IF Expression THEN IfStatements ELSE IfStatements {
}
\end{verbatim}

\subsection{For}

A for loop is constructed with an initializer, a condition and a step expression. It allows for controlled loops.

The syntax: \begin{verbatim}
for Initializer ; Condition ; Step do {
	Code
}
\end{verbatim}

It is defined in the grammar by \begin{verbatim}
FOR Expression ';' Expression ';' Expression DO IfStatements {
}
\end{verbatim}

\subsection{While}

A while loop is given a boolean expression and will continue to execute a given piece of code until that expression evaluates to false.

The Syntax: \begin{verbatim}
while Expression do {
}
\end{verbatim}

It is defined in the grammar by \begin{verbatim}
WHILE Expression DO LBRACKET Statements RBRACKET {
} 
\end{verbatim}

\section{Expression Statements}

An expression statement is a statement such as console.Log("Hello World"); or j := 5 * 4;

The syntax: \begin{verbatim}Expression;\end{verbatim}

It is defined in the grammar as \begin{verbatim}Expression ';' {}\end{verbatim}

\section{Return Statements}

The return statement exits the current function and returns a given expression value. 

It is defined in the grammar as
\begin{verbatim}
'return' ';' | 'return' Expression ';'
\end{verbatim}

\chapter{Functions}

A function is declared using the syntax

\begin{verbatim}
function Name( One or more arguments ) : Type {
 Statements
}
\end{verbatim}

If type is not set then the function is assumed to be of a Void type. A function of Void type does not return any values.

Arguments are written in the form Name : Type, Name : Type, Name : Type so to declare a function which takes two integers as arguments you would write

\begin{verbatim}
func TestArgs(A : int, B : int) {
}
\end{verbatim}
The order in which a function is defined does not matter. Unlike languages such as C you are allowed to call a function before it is defined. This allows mutual recursion as shown in the example

\begin{verbatim}
func even(x : int) : bool {
 if x = 0 then 
   return true;
 return odd(x-1);
}

func odd(x : int) : bool {
 if x = 0 then 
  return false;
 return even(x-1);
}
\end{verbatim}

Multiple functions can share the same name as long as they take different arguments and have the same return type. An example of this is the system String function which has two version one which takes a boolean and one which takes a number. A practical example of this would be

\begin{verbatim}

func PrintTrue(i : bool) {

 if i then 
  sys.Write("True"); 
 else 
  sys.Write("False");

}

func PrintTrue(i : int) {

 if i = 1 then
  sys.Write("True");
 else
  sys.Write("False");

}

\end{verbatim}

this would compile successfully and upon any call to PrintTrue the compiler would resolve the correct function to be executed.

Functions are defined in the grammar as 

\begin{verbatim}

/**
 * Accept zero or more argument definitions
 */

OptionalArgumentDefinitions: {
	} | ArgumentDefinitions {
	}
;

/**
 * Accept one or more argument definitions in the form Name : Type, Name : Type..
 */

ArgumentDefinitions: ArgumentDefinition {
	} | ArgumentDefinitions COMMA ArgumentDefinition {
	}
;

/**
 * The definition of a function. func Name ( Arguments ) { Code } 
 * defines a function of void type and func Name ( Arguments ) : Type { Code }
 * defines a function of a specific type.
 */

Function: FUNCTION WORD LPAREN OptionalArgumentDefinitions 
	RPAREN COLON Type LBRACKET Statements RBRACKET { 	} 
	| FUNCTION WORD LPAREN OptionalArgumentDefinitions 
	RPAREN LBRACKET Statements RBRACKET {
	}
;
\end{verbatim}

\chapter{Packages \& Importing}

Every file in Scribble is seen as a separate package. These packages contain sets of functions and structures which can be used to perform some function, for example one package may contain a linked list structure and a set of functions for inserting and removing elements from said list. Scribble also defines syntax so that packages can import and use structures and functions from other packages.

\subsection{Importing Packages}

One package can be used from another by using the
\begin{verbatim}
package LocalName := import("Package/Path");
\end{verbatim}
syntax.

This syntax allows you to import a package from a given path and give it a local identifier which can be used to reference it. The local identifier allows the avoidance of package naming conflicts, as a importing package can give two packages of the same name ( Such as /tests/math.sc and /math/math.sc ) different local names to be identified by.

To use a function or structure from a different package you first write the packages local name then a '.' and then the name of the function or structure you wish to use. For example to print some output to the screen you write.

\begin{verbatim}
console.Write("Hello World");
\end{verbatim}


\chapter{Standard Functions}

%%%% ADD YOUR BIBLIOGRAPHY HERE
\newpage
\begin{thebibliography}{99}
\addcontentsline{toc}{chapter}{Bibliography}
\bibitem{BISON} Bison - The GNU Parser generator website http://www.gnu.org/software/bison/
\bibitem{FLEX} Flex - The fast lexical analyser website http://flex.sourceforge.net/
\end{thebibliography}
\label{endpage}

\end{document}

\end{article}
